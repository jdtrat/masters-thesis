
\begin{acknowledgements}

\begin{quote}
I dedicate my thesis to both my mom, Joanne Fink, and grandpa, Dr. Gordon Fink. I would not be the person I am without their love and guidance. Plus, they are excellent proof-readers, which I always appreciate! I love you both.
\end{quote}

In May 2022, I will graduate with a Master of Science Degree in Neuroscience from \href{https://wakehealth.edu}{Wake Forest School of Medicine}. This document is the embodiment of my master's thesis. None of this would be possible if not for the incredible support of my mentors, family, and friends.

First and foremost, I'd like to thank Ken Kishida. I've been conducting research with him since February 2018 and am incredibly appreciative of his encouragement and support. Dr.~Kishida has given me the space to follow my interests with programming and neuroscience especially, and this document is the embodiment of that. Within the Kishida Lab, conversations with L. Paul Sands greatly informed my modeling techniques, for which I am appreciative. Emily DiMarco, Rachel Jones, Angela Jiang, and Brittany Liebenow provided feedback and guidance as I developed my thesis.

As an undergraduate, Lucy McGowan, S. Mason Garrison, and Staci Hepler had an immeasurable impact on how I understand, and conduct, scientific programming, Bayesian inference, and data analysis. Katy Lack, Melissa Maffeo, Wayne Pratt, and Terry Blumenthal fostered my interest in Neuroscience. I am extremely grateful to have learned from them.

I'd next like to thank my committee members, Drs. Christian Waugh and Todd McFall (and Ken Kishida!) for their support and guidance as I've worked on my thesis. Their input has made it that much better. Also, \href{http://haines-lab.com}{Nathaniel Haines}'s publications greatly improved my understanding of the underlying mathematics behind computational modeling and he provided invaluable feedback on my work.

To my family, your love and support are everything. Thank you.

\end{acknowledgements}

\tableofcontents

\newpage

\listoffigures

\newpage

\begin{abbreviations}
\begin{description}
    \item \textbf{CPUT} Counterfactual Predicted Utility Theory
    \item \textbf{EUT} Expected Utility Theory
    \item \textbf{EV} Expected Value
    \item \textbf{TDRL} Temporal-Difference Reinforcement Learning
    \item \textbf{RPE} Reward Prediction Error
    \item \textbf{CPE} Counterfactual Prediction Error
    \item $\mathbf{U_C}$ Counterfactual Utility
    \item $\mathbf{V_C}$ Estimated Counterfactual Utility Value
    \item $\mathbf{U_E}$ Expected Utility
    \item $\mathbf{V_E}$ Estimated Expected Utility Value
    \item \textbf{MCMC} Markov Chain Monte Carlo
    \item \textbf{ELPD}  Expected Log Pointwise Predictive Density
    \item \textbf{LOOIC} Leave-One-Out Information Criterion
\end{description}
\end{abbreviations}

\begin{abstract}
In this thesis, I combine insights from economics, pyschology, and neuroscience to develop 'Counterfactual Predicted Utility Theory' (CPUT) as a neurobiologically-plausible theory of decision-making under risk. CPUT is inspired by the observation that sub-second fluctuations in the levels of the neurotransmitter dopamine seemingly reflect factual and counterfactual information. I propose that people incorporate anticipated counterfactual events when making risky decisions. This leads to behavior that is considered ‘irrational’ from a classical economic perspective as described by Expected Utility Theory (EUT). To assess counterfactual predicted utility theory's validity as a theory of decision-making under risk, I compared variations of CPUT and EUT on human choice data from a sure-bet or gamble task using hierarchical Bayesian modeling techniques. I quantified model fit with multiple methods. This includes comparing marginal likelihood model evidence and leave-one-out cross validation predictive accuracy. Compared to EUT, I found that CPUT does not offer a better explanation for the data collected, nor does it generalize as well. However, the sure-bet or gamble task limits our analysis to the gain domain, decreasing the range of counterfactual information available. Further research into the neurobiological mechanisms for processing factual and counterfactual information and the downstream behavioral consequences is warranted.
\end{abstract}
